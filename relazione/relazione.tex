\documentclass[]{article}
\usepackage[a4paper, total={6.5in, 10.5in}]{geometry}%6
\usepackage[ruled,vlined,linesnumbered]{algorithm2e}
\usepackage{listings}
\usepackage{xcolor}
\usepackage{longtable}
\usepackage{graphicx}
\usepackage{amsmath}
\usepackage{float}

\definecolor{codegreen}{rgb}{0,0.6,0}
\definecolor{codegray}{rgb}{0.5,0.5,0.5}
\definecolor{codepurple}{rgb}{0.58,0,0.82}
\definecolor{backcolour}{rgb}{0.95,0.95,0.92}

\lstdefinestyle{mystyle}{
	backgroundcolor=\color{backcolour},   
	commentstyle=\color{codegreen},
	keywordstyle=\color{magenta},
	numberstyle=\tiny\color{codegray},
	stringstyle=\color{codepurple},
	basicstyle=\ttfamily\footnotesize,
	breakatwhitespace=false,         
	breaklines=true,                 
	captionpos=b,                    
	keepspaces=true,                 
	numbers=left,                    
	numbersep=5pt,                  
	showspaces=false,                
	showstringspaces=false,
	showtabs=false,                  
	tabsize=2,
	frame=single
}

% Title Page
\title{Homework2, Algoritmi su Grafi}
\author{Enrico Cancelli, \textit{matr.} 1237293\\
	Alessandro Pegoraro, \textit{matr.} 1240466}

\begin{document}

\maketitle

\begin{abstract}

\end{abstract}

\section{Introduzione}
Lo scopo di questo  progetto è l'analisi e l'implementazione di algoritmi per la risoluzione del problema del \textit{Traveling Salesman} (in seguito TSP) su grafi completi i cui archi hanno pesi che rispettano la disuguaglianza triangolare (anche detto \textit{Triangle-TSP}).
Gli algoritmi implementati sono:
\begin{enumerate}
	\item Algoritmo (esatto) di Held e Karp
	\item Algoritmo (approssimato) di 2-approssimazione basato su MST
	\item Algoritmo (approssimato) basato su euristica costruttiva \textit{cheapest-insertion}
\end{enumerate}
\subsection{Pseudocodice}
Come riferimento per l'implementazione di questi algoritmi è stato utilizzato il seguente pseudocodice spiegato durante le lezioni di laboratorio:\\
\begin{algorithm}[H]
	\SetAlgoLined
	\DontPrintSemicolon
	\KwIn{Graph G, vertex s}
	\KwResult{Hamiltonial cycle}
	//hello
	\caption{Held e Karp}
\end{algorithm}
\begin{algorithm}[H]
	\SetAlgoLined
	\DontPrintSemicolon
	\KwIn{Graph G, vertex s}
	\KwResult{Hamiltonial cycle}
	//hello
	\caption{2-approssimazione}
\end{algorithm}
\begin{algorithm}[H]
	\SetAlgoLined
	\DontPrintSemicolon
	\KwIn{Graph G, vertex s}
	\KwResult{Hamiltonial cycle}
	//hello
	\caption{Cheapest-insertion}
\end{algorithm}
\section{Implementazione}
In questa sezione verranno esposte e adeguatamente motivate le scelte implementative adottate durante lo sviluppo. L'intero progetto è stato realizzato facendo il più possibile uso di codice generico (template di classe per le strutture dati di supporto e template di funzione per gli algoritmi).\\
Infine verrà data una spiegazione dettagliata sulla struttura del codice realizzato ed eventuali note per la compilazione.
\subsection{Parser}
Il dataset fornito è costituito, come per il progetto precedente, da un file per ogni grafo.\\
La struttura dei file è espressa dalla seguente formula BNF:
\begin{verbatim}
<file> :: = <name-dec>
            <type-comment block>
            <dimension>
            <edge-t-dec>
            <edge-f-dec>?
            <display-dec>?
            <coord-sec>
            EOF
<name-dec> :: = NAME: <word> \n
<type-comment block> :: = <type-dec> <comment> | 
                          <comment> <type-dec>
<type-dec> :: = TYPE: TSP \n
<comment> :: = COMMENT: <phrase> \n
<dimension> :: = DIMENSION: <integer> \n
<edge-t-dec> :: = EDGE_WEIGHT_TYPE: <edge-type> \n
<edge-type> :: = GEO | EUC_2D
<edge-f-dec> :: = EDGE_WEIGHT_FORMAT: FUNCTION \n
<display-dec> :: = DISPLAY_DATA_TYPE: COORD_DISPLAY \n
<coord-sec> :: = NODE_COORD_SECTION \n
                 <coordinates>
<coordinates> :: = <coordinate> \n
                   <coordinates> |
                   <coordinate> \n
<coordinate> :: = <integer> <float> <float>
\end{verbatim}
La classe \textit{Parser} si occupa della decodifica di questi file e organizza le informazioni rilevanti quali tipologia delle coordinate, dimensione e coordinate dei nodi creando un oggetto di classe \textit{Graph\_data}.
\subsection{Strutture dati generiche}
\subsubsection{Matrix}
La classe \textit{Matrix$<$T$>$} rappresenta una generica matrice rettangolare di oggetti di tipo T.\\
Essa è utilizzata per la memorizzazione della matrice di adiacenza associata a un grafo.
\subsubsection{MinHeap}
La struttura dati \textit{MinHeap} è la stessa utilizzata nel progetto precedente ed è utilizzata per l'esecuzione dell'algoritmo di Prim all'interno dell'algoritmo di 2-approssimazione.
\subsection{Strutture per la rappresentazione di grafi e sottoinsiemi di nodi}
\subsubsection{Graph\_Data}
Per rappresentare i dati estratti da un file relativo ad un certo grafo, abbiamo usato il template di classe \textit{Graph\_data$<$T$>$}. Gli oggetti di questo tipo contengono i seguenti campi dati:
\begin{itemize}
	\item Nome del grafo
	\item Tipo di coordinate (\verb|cartesian| o \verb|geo|)
	\item Dimensione del grafo (numero di nodi)
	\item Lista di coordinate (rappresentata come un vettore di coppie di elementi di tipo T)
\end{itemize}
Questa classe espone un unico metodo \verb|get_weights| che restituisce la matrice di adiacenza associata al grafo rappresentato dall'oggetto su cui viene invocato. La costruzione di tale matrice dipende dalla tipologia di coordinate.\\
Generalmente la matrice di adiacenza è espressa dalla seguente formula:
$$w[i,j]=dist\_fun(i, j)$$
Dove \verb|dist_fun(i,j)| è la distanza tra il nodo $i$ e $j$ del grafo.\\
In caso le coordinate di un nodo siano di tipo cartesiano, \verb|dist_fun| è la distanza euclidea:
$$dist\_fun(i,j)=round(\sqrt{((i.x - j.x)^2 + (i.y - j.y)^2)})$$
In caso le coordinate di un nodo siano coordinate geografiche, \verb|dist_fun| è la seguente funzione:
$$**formula**$$
\subsubsection{SubSet}

\subsection{Algoritmi}
\subsubsection{Prim}
\subsubsection{Held e Karp}
\subsubsection{2-approssimazione}
\subsubsection{Euristica cheapest-insertion}

\section{Analisi}
%TODO la tabella esce dai margini
\begin{longtable}{|c|c|c|c|c|c|c|c|c|c|}
\hline
\textbf{}        & \multicolumn{3}{c|}{\textbf{Held-Karp (t\_out 20 min)}}                   & \multicolumn{3}{c|}{\textbf{Cheapest Insertion}}          & \multicolumn{3}{c|}{\textbf{2-approssimato}}              \\ \cline{2-10} 
\textbf{Istanza} & \textbf{Soluzione} & \textbf{Tempo} & \textbf{Errore} & \textbf{Soluzione} & \textbf{Tempo} & \textbf{Errore} & \textbf{Soluzione} & \textbf{Tempo} & \textbf{Errore} \\ \hline
burma14      & 3323               & 2.44453            & 0.00\%          & 3588               & 0.0000191          & 0.26\%          & 4003               & 0.0000364          & 0.20\%          \\ \hline
ulysses16    & 6859               & 14.6475            & 0.00\%          & 7368               & 0.0000253          & 0.21\%          & 7788               & 0.0000413          & 0.13\%          \\ \hline
ulysses22    & 7188               & 1202.143           & 0.02\%          & 7709               & 0.0000539          & 0.12\%          & 8308               & 0.0000877          & 0.18\%          \\ \hline
eil51        & 1072               & 1200.662           & 1.52\%          & 494                & 0.0006023          & 0.34\%          & 605                & 0.000194           & 0.42\%          \\ \hline
berlin52     & 18920              & 1200.647           & 1.51\%          & 9004              & 0.0006005          & 0.57\%          & 10402              & 0.0003238          & 0.38\%          \\ \hline
kroA100      & 169787             & 1200.243           & 6.97\%          & 24942              & 0.003957           & 0.43\%          & 30516              & 0.0006457          & 0.43\%          \\ \hline
kroD100      & 150572             & 1200.235           & 6.07\%          & 25204              & 0.0047127          & 0.37\%          & 28599              & 0.0006957          & 0.34\%          \\ \hline
ch150        & 49109              & 1200.121           & 6.52\%          & 7998               & 0.0129079          & 0.49\%          & 9315               & 0.0012785          & 0.43\%          \\ \hline
gr202        & 56533              & 1200.071           & 0.41\%          & 46480              & 0.032091           & 0.29\%          & 52615              & 0.0023568          & 0.31\%          \\ \hline
gr229        & 176922             & 1200.068           & 0.31\%          & 153896             & 0.0464426          & 0.28\%          & 179335             & 0.0031117          & 0.33\%          \\ \hline
pcb442       & 215105             & 1200.188           & 3.23\%          & 60834              & 0.355548           & 0.36\%          & 71264              & 0.0097613          & 0.40\%          \\ \hline
d493         & 112313             & 1200.234           & 2.21\%          & 39969              & 0.501974           & 0.29\%          & 45334              & 0.0128767          & 0.29\%          \\ \hline
dsj1000      & 555573000          & 1201.523           & 28.77\%         & 22291200           & 4.6643             & 0.52\%          & 25526000           & 0.048772           & 0.37\%          \\ \hline
\caption{\textbf{Soluzione e Errore Relativo per ogni Algoritmo}}
\end{longtable}

\end{document}
