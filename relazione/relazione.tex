\documentclass[]{article}
\usepackage[a4paper, total={6.5in, 10.5in}]{geometry}%6
\usepackage[ruled,vlined,linesnumbered]{algorithm2e}
\usepackage{listings}
\usepackage{xcolor}
\usepackage{longtable}
\usepackage{graphicx}
\usepackage{amsmath}
\usepackage{float}

\definecolor{codegreen}{rgb}{0,0.6,0}
\definecolor{codegray}{rgb}{0.5,0.5,0.5}
\definecolor{codepurple}{rgb}{0.58,0,0.82}
\definecolor{backcolour}{rgb}{0.95,0.95,0.92}

\lstdefinestyle{mystyle}{
	backgroundcolor=\color{backcolour},   
	commentstyle=\color{codegreen},
	keywordstyle=\color{magenta},
	numberstyle=\tiny\color{codegray},
	stringstyle=\color{codepurple},
	basicstyle=\ttfamily\footnotesize,
	breakatwhitespace=false,         
	breaklines=true,                 
	captionpos=b,                    
	keepspaces=true,                 
	numbers=left,                    
	numbersep=5pt,                  
	showspaces=false,                
	showstringspaces=false,
	showtabs=false,                  
	tabsize=2,
	frame=single
}

% Title Page
\title{Homework2, Algoritmi su Grafi}
\author{Enrico Cancelli, \textit{matr.} 1237293\\
	Alessandro Pegoraro, \textit{matr.} 1240466}

\begin{document}

\maketitle

\begin{abstract}

\end{abstract}

\section{Introduzione}
Lo scopo di questo  progetto è l'analisi e l'implementazione di algoritmi per la risoluzione del problema del \textit{Traveling Salesman} (in seguito TSP) su grafi completi i cui archi hanno pesi che rispettano la disuguaglianza triangolare (anche detto \textit{Triangle-TSP}).
Gli algoritmi implementati sono:
\begin{enumerate}
	\item Algoritmo (esatto) di Held e Karp
	\item Algoritmo (approssimato) di 2-approssimazione basato su MST
	\item Algoritmo (approssimato) basato su euristica costruttiva \textit{cheapest-insertion}
\end{enumerate}
\subsection{Pseudocodice}
Come riferimento per l'implementazione di questi algoritmi è stato utilizzato il seguente pseudocodice spiegato durante le lezioni di laboratorio:\\
\begin{algorithm}[H]
	\SetAlgoLined
	\DontPrintSemicolon
	\KwIn{Graph G, vertex s}
	\KwResult{Hamiltonial cycle}
	//hello
	\caption{Held e Karp}
\end{algorithm}
\begin{algorithm}[H]
	\SetAlgoLined
	\DontPrintSemicolon
	\KwIn{Graph G, vertex s}
	\KwResult{Hamiltonial cycle}
	//hello
	\caption{2-approssimazione}
\end{algorithm}
\begin{algorithm}[H]
	\SetAlgoLined
	\DontPrintSemicolon
	\KwIn{Graph G, vertex s}
	\KwResult{Hamiltonial cycle}
	//hello
	\caption{Cheapest-insertion}
\end{algorithm}
\section{Implementazione}
In questa sezione verranno esposte e adeguatamente motivate le scelte implementative adottate durante lo sviluppo. L'intero progetto è stato realizzato facendo il più possibile uso di codice generico (template di classe per le strutture dati di supporto e template di funzione per gli algoritmi).\\
Infine verrà data una spiegazione dettagliata sulla struttura del codice realizzato ed eventuali note per la compilazione.
\subsection{Parser}
Il dataset fornito è costituito, come per il progetto precedente, da un file per ogni grafo.\\
La struttura dei file è espressa dalla seguente formula BNF:
\begin{verbatim}
<file> :: = <name-dec>
            <type-comment block>
            <dimension>
            <edge-t-dec>
            <edge-f-dec>?
            <display-dec>?
            <coord-sec>
            EOF
<name-dec> :: = NAME: <word> \n
<type-comment block> :: = <type-dec> <comment> | 
                          <comment> <type-dec>
<type-dec> :: = TYPE: TSP \n
<comment> :: = COMMENT: <phrase> \n
<dimension> :: = DIMENSION: <integer> \n
<edge-t-dec> :: = EDGE_WEIGHT_TYPE: <edge-type> \n
<edge-type> :: = GEO | EUC_2D
<edge-f-dec> :: = EDGE_WEIGHT_FORMAT: FUNCTION \n
<display-dec> :: = DISPLAY_DATA_TYPE: COORD_DISPLAY \n
<coord-sec> :: = NODE_COORD_SECTION \n
                 <coordinates>
<coordinates> :: = <coordinate> \n
                   <coordinates> |
                   <coordinate> \n
<coordinate> :: = <integer> <float> <float>
\end{verbatim}
\subsection{Strutture dati generiche}
\subsubsection{Matrix}
\subsubsection{MinHeap}

\subsection{Strutture per la rappresentazione di grafi e sottoinsiemi di nodi}
\subsubsection{Graph\_Data}
\subsubsection{SubSet}

\subsection{Algoritmi}
\subsubsection{Prim}
\subsubsection{Held e Karp}
\subsubsection{2-approssimazione}
\subsubsection{Euristica cheapest-insertion}

\section{Analisi}

\end{document}
